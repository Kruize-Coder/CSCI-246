\documentclass[11pt]{article}

%Don't change any thing before \begin{document}
%They are not useful for now, but later when you try to add figures
%these might be useful. In fact if you use sth fancy, you might need
%to add more packages, or macros.
\usepackage{amssymb,amsmath}
\usepackage{times,psfrag,epsf,epsfig,graphics,graphicx}
\usepackage{algorithm}
\usepackage{algorithmic}

\begin{document}
\date{October 12th 2020}

\title{CSCI 246: Assignment~4~(5 points)}

\author{Kruize, Zach, Kai}

\maketitle


\noindent
This assignment is due on {\bf Monday, Oct 12, 8:30pm}. It is strongly
encouraged that you use Latex to generate a single pdf file and upload it
under {\em Assignment 4} on D2L. But there will NOT be a penalty for not
using Latex (to finish the assignment). This could be a group-assignment,
so you could form a group with $\leq 3$ students (mathematically, this means
you can also do it by yourself).
\newline

\section*{Problem 1.}

\noindent
Use the test for primality to determine whether the following numbers are
prime or not; if not, list one factor greater than one.

{\bf Test for Primality:} ~~ Given an integer $n>1$, to test whether $n$ is
prime check to see if it is divisible by a prime number less than or equal to
$\sqrt{n}$. If it is not divisible by any of these numbers, then it is prime.
\newline
\newline
Largest number is 9,127 so $\sqrt{9127} < 96$
\newline
All prime numbers between 1 and 96 are 2, 3, 5, 7, 11, 13, 17, 19, 23, 29, 31, 37, 41, 43, 47, 53, 59, 61, 67, 71, 73, 79, 83, 89,
\newline

\noindent
(1.1) 5,369.
\newline
\newline
5369/7 = 767, therefore it is not prime
\newline

\noindent
(1.2) 9,127.
\newline
\newline
9127 is prime
\newline

\noindent
\newline
\newline
(1.3) 3,927.
\newline
\newline
3927/3 = 1309 therefore it is not prime.
\newline

\noindent
(1.4) 8,647.
\newline
\newline
8647 is prime


\section*{Problem 2.}

\noindent
Compute the following, assume that $n>4$.
$$\frac{(n-1)!}{(n-5)!} = $$ 

%\noindent
{\bf Answer:}~~ $(n-1)(n-2)(n-3)(n-4)$
%\newline
\newpage

\section*{Problem 3.}

Prove the following statement by mathematical induction.

For every integer $n\geq 0$, $\sum\limits_{i=1}^{n+1}i\cdot 2^i=n\cdot 2^{n+2}+2$.
\newline
\newline
P(0) =  $\sum\limits_{i=1}^{n+1}i\cdot 2^i=0\cdot 2^{0+2}+2$
\newline
If P(k) is true than P(k+1) is also true
\newline
Left hand side is $\sum\limits_{i=1}^{k+1} i\cdot  2^i$
\newline
Right hand side of P(k) is $k \cdot 2^{k+2} + 2$
\newline
\newline
Left side of P(k+1) is then $\sum\limits_{i=1}^{k+1} i \cdot 2^i + (k+2)\cdot2^{k+2}$
\newline
\newline
Right side is then $(k+1)\cdot 2^{(k+3)}+2$
\newline
\newline
Then substitute right side into left side
\newline
\newline
$k\cdot 2^{k+2}+2 + (k+2)\cdot 2^{k+2}$
\newline
\newline
Simplifies down to
\newline
\newline
$=(k+1)\cdot 2^{(k+3)}+2$
\newline
\newline
Therefore P(k+1) is true which proves that P(k) is also true through induction.


%\noindent
%{\bf Proof:}~~ .........
%\newline

\newpage

\noindent
\section*{Problem 4.}

Use the formula for the sum of a geometric sequence to write the following sum in a closed form.
\newline

$8+8^2+8^3+\cdots 8^k$, where $k\geq 1$ is an integer.
\newline
\newline
$a\frac{r^k-1}{r-1} = 8 \cdot \frac{8^k-1}{8-1} = 8 \cdot \frac{8^k-1}{7} = \frac{8}{7}(8^k-1)$
\newline
\newline
Closed form sum: $\frac{8}{7}(8^k-1)$

%\noindent
%{\bf Answer:}~~ ...........
%\newline
\newpage


\section*{Problem 5.}

Prove the following statement by mathematical induction.
\newline

For every integer $n\geq 0$, $2^n<(n+2)!$.
\newline
\newline
P(n) = $2^n<(n+2)!$
\newline
Left hand side of P(0) = $2^0 = 1$
\newline
Right side is $(0+2)! = 2! = 2$
and 1 $<$ 2 therefore P(0) is true
\newline
\newline
If P(k) is true than P(k+1) is also true and vice versa
\newline
\newline
Suppose P(k) is true ie. $2^k < (k+2)!$
\newline
If we want P(k+1) to be true, then $((k+1)+2)! = (k+3)! = (k+3)(k+2)! > (k+3)\cdot 2^k$
\newline
$k+3 > 2$ whenever $k\geq0$
\newline
Therefore $((k+1)+2)! > 2^{k+1}$
\newline
That means that P(k+1) is true
\newline
\newline
By mathematical induction P(n) is true for all $n \geq 0$
\newline
Or for every integer $n \geq 0$, $n^2 < (n+2)!$


%\noindent
%{\bf Proof:}~~ .................
%\newline
%\newline

\end{document}