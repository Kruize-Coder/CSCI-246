\documentclass[11pt]{article}

%Don't change any thing before \begin{document}
%They are not useful for now, but later when you try to add figures
%these might be useful. In fact if you use sth fancy, you might need
%to add more packages, or macros.
\usepackage{amssymb,amsmath}
\usepackage{times,psfrag,epsf,epsfig,graphics,graphicx}
\usepackage{algorithm}
\usepackage{algorithmic}
\usepackage{xcolor}

\begin{document}
\date{}

\title{CSCI 246: Assignment~8~(6 points)}

%\author{Your Name Here}

\maketitle


\noindent
This assignment is due on {\bf Thursday, Nov 12, 9:30pm}. It is strongly
encouraged that you use Latex to generate a single pdf file and upload it
under {\em Assignment 8} on D2L. But there will NOT be a penalty for not
using Latex (to finish the assignment). This could be a group-assignment,
so you could form a group with $\leq 3$ students (mathematically, this means
you can also do it by yourself).
\newline
 
\section*{Problem 1.}

\noindent
A company offers a raffle whose grand prize is a \$60,000 new car. Additional prizes are a \$1,000 television and a \$500 computer. Tickets cost \$20 each. Ticket income will be donated to a charity. If 3,000 tickets are sold, what is the expected gain or loss (in dollars) of each ticket? (Round your answer to the nearest cent.)
\newline
\newline
There are three prizes. The probability of winning each of them is 1/3000.
\newline
$= 3500(\frac{1}{3000}) + 900(\frac{1}{3000}) + 600(\frac{1}{3000})$
\newline
$= 12.2$ 
\newline
12.2 - 20 = -7.80 loss expected
\newline


\newpage

\section*{Problem 2.}

\noindent
One urn contains {\color{red}9} blue balls and {\color{red}12} white balls, and a second urn contains {\color{red}12} blue balls and {\color{red}8} white balls. An urn is selected at random, and a ball is then randomly chosen from the
urn. (Round your answers to one decimal place.)
\newline

\noindent
(2.1) What is the probability (as a \%) that the chosen ball is blue?
\newline
\newline
\newline
$\frac{1}{2} * \frac{9}{21} + \frac{1}{2 * \frac{19}{20}} = 0.5143$
\newline
= 51.43 percent that ball is blue
\newline
\newline
\newline
\newline

\noindent
(2.2) If the chosen ball is blue, what is the probability (as a \%) that it
came from the first urn?
\newline
\newline
\newline
$\frac{\frac{1}{2} * \frac{9}{21}}{\frac{1}{2} * \frac{9}{21} + \frac{1}{2} * \frac{12}{20}}$
\newline
\newline
$= \frac{\frac{9}{21}}{\frac{9}{21} + \frac{12}{20}} + \frac{\frac{9}{21}}{180+252}$
\newline
\newline
$= 0.4167 * 100$
\newline
\newline
= 41.67 percent that the blue ball came from the first urn.
\newpage


\section*{Problem 3.}

\noindent
Prove that at a party with at least two people, there are at least two mutual
acquaintances or at least two mutual strangers.
\newline
\newline
Let there be N people in a group. A person can only have N-1 friends. Look at the contrary that all N people have a different number of friends. Sequence: 0,1,2,3,....,N-1, There also must be a person with exactly N-1 friends.
\newline
\newline
There must be at least one with N-1 friends. But if that is true, then all others have this person as a friend which implies that there is no one with no friends. Therefore, the only possible number comes from 1,2,3,...,N-1.
\newline
\newline
There are at least two people with the same number of friends, so it must be true. 
\newline
\newline
Therefore in a party with at least two people, there are at least two mutual acquaintances or at least two mutual strangers.
\newpage

\section*{Problem 4.}

List the adjacency matrices for the following two graphs.
\newline

\begin{figure}[htbp]
\begin{center}
\includegraphics[scale=0.70]{fig1.pdf}
\end{center}
\label{fig1}
\caption{\bf Two graphs $G_1$ and $G_2$.}
\end{figure}
%
%\noindent
%{\bf Answer:}~~.
%\newline
\newpage

\section*{Problem 5.}

Consider the following description of a graph:
\newline

  Graph, circuit-free, seven vertices, four edges.
\newline

\noindent
Can such a graph exist? If so, draw an example. If not, explain the reason.
\newline
\newline
\newline
\newline
%\noindent
{\bf Answer:}Yes
%\newline
%\newline

\end{document}